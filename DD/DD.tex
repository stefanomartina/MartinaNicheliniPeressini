\documentclass{article}
\usepackage{graphicx}
\usepackage{hyperref}
\hypersetup{
    colorlinks,
    citecolor=black,
    filecolor=black,
    linkcolor=black,
    urlcolor=black
}

\begin{document}

	\begin{figure}[t]
	\centering
	\includegraphics[height=6.25cm,keepaspectratio]{Figures/logo}
	\end{figure}
	
	\title{TrackMe \\ Software Engineering 2 Project \\ 
			\textit{DD Document} }
	\author{Stefano Martina, Alessandro Nichelini, Francesco Peressini
		\\ \\ A.Y. 2018/2019 \\ Version 1.0.0}
		
\maketitle
\newpage

\tableofcontents

\newpage
\section{Introduction}

\subsection{Purpose}
\subsection{Scope}
\subsection{Definitions, Acronyms, Abbreviations}
\subsection{Revision history}
\subsection{Reference Documents}
\subsection{Document Structure}
\section{Architectural design}

The system is going to be implemented with a three tier architecture. 

\subsection{Overview: High-level}
\subsection{Component view}
\subsection{Deployment view}
\subsection{Runtime view}
\subsection{Component interfaces}
\subsubsection{API structure}

All the api system will be implemented referring to a single endpoint www.data4help.cloud. Users' applications and third parties will refer to different subdomain:

\begin{itemize}
	\item \textit{www.data4help.cloud/api/users} will be the specific endpoint for the application that serves users.
	\item \textit{www.data4help.cloud/api/thirdparties} will be the specific endpoint for thirdparties.
\end{itemize}


\subsection{Selected architectural styles and patterns}
\subsection{Other design decision}

\section{User interface design}

\section{Requirements traceability}

\section{Implementation, integration and test plan}
 
\section{Effort spent}

\section{References}
\newpage


 
\end{document}