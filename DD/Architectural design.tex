\documentclass[DD.tex]{subfiles}
\begin{document}
\section{Architectural design}
\subsection{Overview: High-level}

The system is going to be implemented with a three tier architecture. Tiers are  as briefly described by the following schemas.

\begin{figure}[h!]
	\centering
	\includegraphics[height=6.00cm,keepaspectratio]{Figures/GeneralSchema}
	\caption{High level view of the system's architecture}
\end{figure}

\begin{figure}[h!]
	\centering
	\includegraphics[height=8.00cm,keepaspectratio]{Figures/ThreeTierSchema}
	\caption{Distribution of application's function among the tier}
\end{figure}

The decision of this kind of architecture has been taken in order to build the system in the most modular possibile way. Here are described layer organisation:

\begin{itemize}
	\item Presentation layer:\begin{itemize}
			\item \textit{Mobile clients}: users will be given with a iOS application which will be a view of the entire system.
			\item \textit{Third parties}: third parties will be given with a light web interfaces to register/manage API access and they will be authorised to communicate with the system.
			\end{itemize}
	\item \textit{Logic layer}: logic layer will implement all the logic of the entire system and will handle communications between clients' app and the data layer.
	\item \textit{Data layer}: data layer will be implemented in third party's cloud system and will keep persistent users' data.
\end{itemize}

The idea is to keep as separate as possibile the logic layer from the data layer in order to let the system grow in a modular fashion and let us change cloud data provider as the system's dimension grow with the minimum effort.
\newpage
\subsection{Component view}

\begin{figure}[h!]
	\centering
	\includegraphics[height=8.00cm,keepaspectratio]{Figures/DeploymentDiagram}
	\caption{Deployment Dyagram}
\end{figure}

The Deployment diagram emphasises the nodes on which the platform runs:\\
\textbf{InHouse} box represent the components developed inside the company.
The components in the \textbf{External box} represent the “on demand” software.\\
\begin{itemize}
\item	\textbf{Smartphone} : the application deployed on the iOs Smartphone used by the user. Users are able to retrieve data from the application server and, also, from the external 			server API directly (e.g. maps by Google Maps)
\item \textbf{Web Browser API Interface} : A web page for Third Parties registration. It provides a secret to do a request to the API Server.
\item \textbf{Application Server }: the  main logic core of the application. It’s the only one access point to the DB but also it communicates with the external server API
\item  \textbf{External Service} : Both the external services are represented as a blackbox because we don’t know how they are implemented 
\end{itemize}




\subsection{Deployment view}
\subsection{Runtime view}
\subsection{Component interfaces}
\subsubsection{API structure}

All the api system will be implemented referring to a single endpoint www.data4help.cloud. Users' applications and third parties will refer to different subdomain:

\begin{itemize}
	\item \textit{www.data4help.cloud/api/users} will be the specific endpoint for the application that serves users.
	\item \textit{www.data4help.cloud/api/thirdparties} will be the specific endpoint for thirdparties.
\end{itemize}


\subsection{Selected architectural styles and patterns}
\subsection{Other design decision}
\end{document}